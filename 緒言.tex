% !TEX root = main.tex
%%%%%%%%%%%%%%%%%%%%%%%%%%%%%%%%%%%%%%%%%%%%%%%%%%%%%%
\section{序論}
%%%%%%%%%%%%%%%%%%%%%%%%%%%%%%%%%%%%%%%%%%%%%%%%%%%%%%
\subsection{始めに}
人工衛星とは,地球の周りをまわっている人工物体のことをいい,搭載しているセンサなどで地球の気象や地上の状況を調べる地球観測衛星,
位置情報を正確に測る測位衛星,インターネットなどを構成する通信衛星などがある.
内閣府の調査によれば,2023年に打ち上げられた人工衛星等の機数は,過去最大の2,901機であり,10年前と比べて約14倍に増加したという.
数十~数千の人工衛星を一体的に運用しネットワークを構築する,メガコンステレーションの構築への取り組みや,
先進国による宇宙利用への期待の高まりから,宇宙輸送のニーズは一層拡大することが見込まれている.

 人工衛星は,大きいものでは例えばISS(国際宇宙ステーション)がある.
それに対し小さいものでは,一辺10cmの立方体サイズの超小型人工衛星が存在している.

 大型人工衛星,中型人工衛星と呼ばれる人工衛星は,多くが国家プロジェクトとして開発され,打ち上げられている.
その特徴として,高機能で信頼性が高く,複雑な用途,複数のミッションに活用できるという利点がある.
しかし,設計・製造に莫大な費用と時間が必要となる.

 それに対し,超小型人工衛星は,機能が制限され,単一のミッションしか実行できないという欠点がある.
その代わり,短期間での開発・低コストでの打ち上げが可能であるため,
学生,大学院生が在学中に開発・打ち上げをすることが可能となっている.
また,世界各国が宇宙開発研究でしのぎを削るなか,新しい技術を素早く試せることも,
注目されている一つの理由である.

超小型人工衛星での姿勢制御は,地球観測や高速通信といったミッションで応用される.

 超小型人工衛星は,その小ささから,搭載可能な機器のサイズや重量に制限がある.
そのため,リアクションホイールやスラスタといった,性能の代わりに大型な機器は搭載が難しい.
そこで,姿勢制御に磁気トルカと呼ばれる電磁石が用いられることが多い.
磁気トルカは,電磁誘導を利用してコイルに発生させた磁場と地磁気を反応させることでトルクを発生させ,回転力を得るものである.


\subsection{本研究の目的}
 人工衛星の姿勢制御を地上で実験し,研究・検討するには,通常球面の空気軸受けを使った3軸テーブルが用いられる.
しかし,3軸で制御する衛星でも,3軸を同時に制御する状態は必要とせず,1軸のみでの制御実験を先に検証する場合が多い.
また,3軸テーブルでの実験の設計の難しさから,1軸テーブルは有効な実験道具となる.

 本研究では,小型衛星の模型を用いて,磁気トルカで衛星の姿勢制御を行う様子を再現し,
人工衛星によく用いられる,B-dot制御則やクロスプロダクト則といった制御理論

