% !TEX root = main.tex
%%%%%%%%%%%%%%%%%%%%%%%%%%%%%%%%%%%%%%%%%%%%%%%%%%%%%%
\section{序論}
%%%%%%%%%%%%%%%%%%%%%%%%%%%%%%%%%%%%%%%%%%%%%%%%%%%%%%
\subsection{始めに}
 人工衛星とは,地球の周りをまわっている人工物体のことをいい,搭載しているセンサなどで地球の気象や地上の状況を調べる地球観測衛星,
位置情報を正確に測る測位衛星,インターネットなどを構成する通信衛星などがある.
内閣府の調査によれば,2023年に打ち上げられた人工衛星等の機数は,過去最大の2,901機であり,10年前と比べて約14倍に増加したという.
数十~数千の人工衛星を一体的に運用しネットワークを構築する,メガコンステレーションの構築への取り組みや,
先進国による宇宙利用への期待の高まりから,宇宙輸送のニーズは一層拡大することが見込まれている.


\subsection{超小型人工衛星とは}
 人工衛星は,大きいものでは例えば大きさが約108.5 m×72.8 m のISS(国際宇宙ステーション)がある.
それに対し小さいものでは,一辺10cmの立方体サイズの超小型人工衛星が存在している.
超小型人工衛星の定義は明確には決まっておらず,全質量が100 kg 以下とするものや,50 kg 以下とするものもある.
本研究では,超小型人工衛星の中でも,10 cm ×10 cm ×10 cm の立方体を1Uとして規格化された小型衛星(CubeSat)に絞って述べる.
大型人工衛星,中型人工衛星と呼ばれる人工衛星は,多くが国家プロジェクトとして開発され,打ち上げられている.
その特徴として,高機能で信頼性が高く,複雑な用途,複数のミッションに活用できるという利点がある.
しかし,設計・製造に莫大な費用と時間が必要となる.
反対に,超小型人工衛星は,機能が制限され,単一のミッションしか実行できないという欠点がある.
その代わり,短期間での開発・低コストでの打ち上げが可能であるため,
学生,大学院生が在学中に開発・打ち上げをすることが可能となっている.
また,世界各国が宇宙開発研究でしのぎを削るなか,新しい技術を素早く試せることも,
注目されている一つの理由である.

% 大きさが伝わりやすい超小型人工衛星の画像を探しています

 人工衛星は,地球との通信のためのアンテナや,発電のための太陽光パネルの向きを制御するために,姿勢の制御が行われる.
人工衛星の姿勢制御の方式には,主に以下のようなものがある.

\begin{itemize}
    \item 重力傾度姿勢安定方式\\
    軌道を廻る衛星が地球の重力により衛星の軸のうち一つが常に地球の中心を向く性質を利用するもの.
    能動的な制御やエネルギーを必要としない.
    \item スピン安定方式\\
    慣性主軸の回りにスピンを与えて,コマのように安定させる方式.衛星全体を回転させるシングルスピン方式と,
    通信用のアンテナを回転させないため,アンテナ部と衛星本体をそれぞれ違う方向に回転させるデュアルスピン方式がある.
    \item 3軸安定方式\\
    衛星の直交する3軸(ロール・ピッチ・ヨー角)の各軸について制御する方式である.
    リアクションホイールを用いるものには,ジャイロ効果を用いるバイアスモーメンタム方式と,
    外乱に応じて回転数を変化させるゼロモーメンタム方式がある.
    磁気トルカによる制御は,3軸安定方式にあたる.
\end{itemize}

 特に3軸安定方式に用いられるアクチュエータには,スラスター,リアクションホイール,
コントロールモーメントジャイロ,磁気トルカといったものがある.
スラスターは,推進剤を噴出し,その反力で推進力を得るものである.
またリアクションホイールは,ホイールを回転させて生ずる反力によるトルクで回転を起こし,
コントロールモーメントジャイロは,その回転の向きを変えることにより姿勢を制御する.
磁気トルカは,電磁誘導を利用してコイルに発生させた磁気モーメントと
地磁気を反応させることでトルクを発生させ,回転力を得るものである.
超小型人工衛星は,その小ささから,搭載可能な機器のサイズや重量に制限がある.
そのため,リアクションホイールやスラスタといった,性能の代わりに大型な機器は搭載が難しい.
そこで,姿勢制御に磁気トルカと呼ばれる電磁石やCMG(コントロールモーメントジャイロ)が用いられることが多い.



\subsection{本研究の目的}
 人工衛星の姿勢制御を地上で実験し,研究・検討するには,通常球面の空気軸受けを使った3軸テーブルが用いられる.
しかし,3軸で制御する衛星でも,3軸を同時に制御する状態は必要とせず,1軸のみでの制御実験を先に検証する場合が多い.
また,3軸テーブルでの実験の設計の難しさから,1軸テーブルは有効な実験道具となる.

 本研究では,小型衛星の模型を用いて,磁気トルカを用いて1軸のみの衛星の姿勢制御を行う様子を再現し,
人工衛星によく用いられる,B-dot制御則やクロスプロダクト則といった制御理論や,P制御,P-D制御といった理論を検証する.
実際の人工衛星の動きを完璧にトレースするのではなく,あくまでもデモンストレーションとして,制御理論の特徴を反映できることを確かめる.