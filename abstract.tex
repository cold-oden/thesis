% !TEX root = main.tex
\begin{center}
\section*{論\,文\,要\,旨}                      %% ここに番号をつけない
\end{center}

 人工衛星とは,惑星である地球の周りをまわっている人工の衛星であり,搭載しているセンサな
どで地球の気象や地表・海面の温度,植物の有無などを調べる地球観測衛星,位置情報を正確に測
る測位衛星,インターネットなどを構成する通信衛星などがある.
内閣府の調査によれば,2023年に打ち上げられた人工衛星等の機数は,過去最大の2,901機であり,10年前と比べて約14倍に増加したという.
超小型人工衛星は,機能が制限され,単一のミッションしか実行できないという欠点がある.
その代わり,短期間での開発・低コストでの打ち上げが可能であるため,
学生が在学中に開発・打ち上げをすることが可能となっている.
超小型人工衛星は,その小ささから,搭載可能な機器のサイズや重量に制限がある.
そのため,リアクションホイールやスラスタといった,性能の代わりに大型な機器は搭載が難しい.
そこで,姿勢制御に磁気トルカやCMG(コントロールモーメントジャイロ)が用いられることが多い.

 本研究では,小型衛星の模型を用いて,磁気トルカを用いて1軸のみの衛星の姿勢制御を行う様子を再現し,
人工衛星によく用いられる,B-dot制御則やクロスプロダクト則といった制御理論や,P制御,PD制御といった基本的な理論を検証する.

 実験の結果,B-dot制御則については,定常偏差と整定時間から,衛星の角速度を0に指向するという,制御則の特徴が現れる結果が得られた.
P,PD制御,クロスプロダクト則については,軸受の摩擦の影響による定常偏差が残り,評価が難しかった.
作製した実験装置においては,整定時間を重視すればB-dot制御則が,定常偏差を重視すればクロスプロダクト則が最も良いと判断できる結果が得られた.
しかし,PD制御,クロスプロダクト則についてはゲイン設定の余地があるため,定常偏差,オーバーシュートの改善がまだ済んでいない状態にある.

 本実験装置を磁気トルカおよび制御理論を検証する実験装置として評価すると,特にB-dot制御則については,角速度を 0 に指向するという B-dot 制御則の
特徴が現れていると考えられ,またPD制御,クロスプロダクト則については,改良の余地はあるものの,コントローラの設計が行えるため,人工衛星の姿勢制御についての簡単な実験装置として用いるこ
とは可能であると考える.

%  実際の宇宙環境と異なる点として,電流値と地磁気に見立てた磁石の磁束密度による磁気トルカに生ずるトルクの違い,
% 回転時の摩擦を軽減する転がり軸受の滑り摩擦,大気の有無による空気抵抗の差が挙げられ,これらを要因とする挙動への影響が生じていることが予想される.

 今後の展望として,実験装置の環境を宇宙空間に近づけること,磁気トルカを増やした際の検証,実際の人工衛星の挙動との比較が挙げられる.
実験装置の環境を宇宙空間に近づけるには,空気抵抗や転がり軸受の摩擦,外部電源の接続といった点を改善する必要がある.
中でも転がり軸受の滑り摩擦の軽減は実行しやすく,また外部電源を実験装置に搭載することも可能であると考えられる.
本研究では磁気トルカを1つのみ用いて検証を行ったが,実際に打ち上げられている超小型人工衛星には,磁気トルカを2つ以上用いるものも多いため,
本実験装置に磁気トルカを2つ搭載した際の挙動の変化も検証する必要がある.
さらに,本実験装置の有用性の判断のため,本実験装置の条件のもとシミュレーションを行い,その結果との定量的な比較を行うことで有用性の定量的な判断ができるようになる可能性がある.

