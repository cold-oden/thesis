% !TEX root = main.tex
%%%%%%%%%%%%%%%%%%%%%%%%%%%%%%%%%%%%%%%%%%%%%%%%%%%%%%%%%%%%%%%%%%%%%%%%
\begin{center}
\section*{\kintou{2.5zw}{謝辞}}                      %% ここに番号をつけない
\vspace*{-2zh}
\end{center}
\addcontentsline{toc}{section}{謝辞} %% 目次に番号をつけない
%%%%%%%%%%%%%%%%%%%%%%%%%%%%%%%%%%%%%%%%%%%%%%%%%%%%%%%%%%%%%%%%%%%%%%%%

 本研究を進めるにあたり,多くの方々にご指導ご鞭撻を賜りました.
指導教員の西佑介教授からは多大なご指導を賜り,数々の助言を頂きました.
また,編入学試験に際して,面接練習や受験勉強の場を提供して頂きました.心より感謝いたします.\\
 電子制御工学科教員の仲川力教授には,ガウスメータを貸与していただき,
考察を深めることができました.心より感謝いたします.\\
 本科5年の野口史遠氏には,Arduinoや回路制作で行き詰まった際に
多くの助言を頂きました.心より感謝いたします.\\
 専攻科1年の三宅航希氏と,本科5年の正木律氏には,特に電子回路について助言をいただき,
回路の改良にご助力頂きました.心より感謝いたします.\\
 本科5年の神田大訓氏には,研究内容は異なるものの,実験に対するひたむきな姿勢に感化されることもありました.
心より感謝いたします.\\
 本科5年の澤田一理氏には,卒研発表のスライドにおいて自分の視点にはない鋭い指摘を頂き,
よりよい発表スライドを作成できました.心より感謝いたします.\\
 本科5年の櫻井蒼真氏には,卒業研究の中間発表の練習に際して,発表に関する改善点を指摘いただき,
満足のいく発表ができました.心より感謝いたします.\\
 本科5年の林巧巳氏には,編入学試験の勉強について,度々議論を交わしていただき,互いに高めあうことができました.
心より感謝いたします.\\
 最後に,研究や日頃の生活において,有形無形の援助をしていただいた舞鶴高専電子制御工学科教員
の皆様,および生活を支えてくださった家族と友人に心より感謝いたします.\\