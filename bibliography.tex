% !TEX root = main.tex
%%%%%%%%%%%%%%%%%%%%%%%%%%%%%%%%%%%%%%%%%%%%%%%%%%%%%%%%%%%%%%%%%%%%%%%%
\begin{center}
\section*{\kintou{5zw}{参考文献}}                      %% ここに番号をつけない
\vspace*{-2zh}
\end{center}
\addcontentsline{toc}{section}{参考文献} %% 目次に番号をつけない
%%%%%%%%%%%%%%%%%%%%%%%%%%%%%%%%%%%%%%%%%%%%%%%%%%%%%%%%%%%%%%%%%%%%%%%%

\begin{thebibliography}{99}
\bibitem{intro1}
	内閣府宇宙開発戦略推進事務局:
	宇宙輸送を取り巻く環境認識と将来像,
	宇宙政策委員会 宇宙輸送小委員会 第2回会合 資料,
	1/4 (2023)

\bibitem{sat}
	JAXA「人工衛星プロジェクト」:
	https://www.satnavi.jaxa.jp/ja/index.html

\bibitem{intro2}
	JAXA「国際宇宙ステーション(ISS)とは」:
	\url{https://humans-in-space.jaxa.jp/iss/about/}

\bibitem{micro}
	テレビ愛知~ザ特集「"超小型"衛星の大きな夢」:
	\url{https://tv-aichi.co.jp/you/2019/03/012464.html}

\bibitem{medium}
	中央日報「韓国、精密地上観測用次世代中型衛星1号を3月に打ち上げ」:
	\url{https://japanese.joins.com/Jarticle/274738}
	
\bibitem{large}
	JAXA「「いぶき2号」の大きさ」:
	\url{https://fanfun.jaxa.jp/eos/topics/post_35}

\bibitem{intro3}
	二宮敬虔,中谷一郎:
	人工衛星の姿勢安定と制御,
	電気学会雑誌,
	94巻, 1号, p.19 (1974)

\bibitem{intro4}
	茂原正道:
	重力傾度を利用する人工衛星の姿勢制御,
	日本航空学会誌,
	15巻, 159号, p.1 (1967)

\bibitem{intro5}
	中丸邦男,田中俊輔:
	人工衛星の姿勢制御,
	計測と制御,
	30巻, 10号,pp.1-2 (1991)

\bibitem{intro6}
	村上力,岡本修,中島厚,木田隆:
	人工衛星の姿勢制御研究用1軸テーブル装置について,
	航空宇宙技術研究所資料,
	257巻, p.19 (1974)

\bibitem{kawata}
	川田昌克:
	MATLAB/Simulinkによる制御工学入門,
	森北出版(2020)

\bibitem{bdot}
	Goddard Space Flight Center:
	Flight Mechanics Symposium 1997: Proceedings of a Conference Sponsored by NASA Goddard
	 Space Flight Center at Goddard Space Flight Center, Greenbelt, Maryland, May 19-21, 1997,pp.79-81
	 IICA Biblioteca Venezuela (1997)

\bibitem{cross}
	Kikuko Miyata and Jozef C. van der Ha:
	ATTITUDE CONTROL BY MAGNETIC TORQUER,pp.1047-1048,
	Advances in the Astronautical Sciences (2009)

\bibitem{torquer}
	多羽田俊,三品博司:超小型人工衛星に関する研究
	―磁気トルカによる姿勢制御システムの開発と熱解析コードの開発―,p.1
	日本大学(2007)

\bibitem{chijiki}
	気象庁:
	https://www.jma.go.jp/jma/press/2405/11a/20240511\_chijiki.html
	
% \bibitem{NIT1}
% 	舞鶴高専ホームページ:
% 	\url{http://www.maizuru-ct.ac.jp}
\end{thebibliography}
% *******************************************